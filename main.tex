% !TEX encoding = UTF-8 Unicode
\documentclass[12pt,portuguese,a4paper,pdftex,openany]{book}
\usepackage{lmodern}			% Usa a fonte Latin Modern
\usepackage[utf8]{inputenc}		% Codificacao do documento (conversão automática
\usepackage[T1]{fontenc}		% Selecao de codigos de fonte. dos acentos)
\usepackage{babel}
\usepackage{microtype} 			% para melhorias de justificação
\usepackage{indentfirst}		% Indenta o primeiro parágrafo de cada seção.
\usepackage{comment}
\usepackage{graphicx}
\usepackage{color,times}
\usepackage{subfigure}
\usepackage{algorithmic}
\usepackage{algorithm2e}
\usepackage{float}
\usepackage{rotating}
\usepackage{amsmath} % amssymb internally loads amsfonts 
\usepackage{bm}
\usepackage{amssymb}
\usepackage{amsthm} % amsthm must be loaded after amsmath
\usepackage[pdftex]{hyperref} % must be loaded after ucs package to avoid conflict
\usepackage{pdfpages}
\usepackage{nicefrac}
\usepackage{fancyhdr}
\usepackage{multirow}
\usepackage{multicol}
% \usepackage{times}
\usepackage{pdflscape}
\usepackage{cancel}
\usepackage{shadow}
\usepackage{longtable}
\usepackage{titlesec}
\usepackage{bigints}
\usepackage[brazilian,hyperpageref]{backref}		% Paginas com as citações na bibl
\usepackage[alf]{abntex2cite}						% Citações padrão ABNT
\usepackage{placeins}								%fixar elementos com o FLOATBARRIER
\usepackage{color,colortbl,multirow}		     	%Colorindo células de tabelas
%\usepackage[table]{xcolor}          				%Colorindo células de tabelas
\usepackage[hypcap,font=footnotesize,labelfont=bf,textfont=bf]{caption}
\usepackage{listings}
\usepackage{xcolor} 
%\usepackage{refcheck}              % Verificando numeração das fórmulas e figuras
% Define cores em RGB
\definecolor{dkgreen}{rgb}{0,0.6,0}
\definecolor{gray}{rgb}{0.5,0.5,0.5}
\definecolor{mauve}{rgb}{0.58,0,0.82}

%%%%%%%%%%%%%%%%%%%%%%%%%%%%%%%%%%%%%%%%%%%%%%%
% Define grifo amarelo |V
%%%%%%%%%%%%%%%%%%%%%%%%%%%%%%%%%%%%%%%%%%%%%%%
\usepackage{soul}

%%%%%%%%%%%%%%%%%%%%%%%%%%%%%%%%%%%%%%%%%%%%%%%
%permite a utilização de links
%%%%%%%%%%%%%%%%%%%%%%%%%%%%%%%%%%%%%%%%%%%%%%%
\usepackage[pdftex]{hyperref}
% \usepackage{pbox}
% \usepackage{parskip}
% \usepackage{showkeys} % help to correct cross-referencing
\usepackage[nottoc]{tocbibind}
\setcounter{tocdepth}{2}
\usepackage[verbose,a4paper,left=3.0cm,right=3.0cm,top=2.5cm,bottom=2.5cm]{geometry}

%%%%%%%%%%%%%%%%%%%%%%%%%%%%%%%%%%%%%%%%%%%%%%%
%numeração das páginas
%%%%%%%%%%%%%%%%%%%%%%%%%%%%%%%%%%%%%%%%%%%%%%%
\usepackage{setspace}
\pagestyle{fancy}
\fancyhf{}
\fancyfoot[C]{\thepage}
\renewcommand{\headrulewidth}{0pt}

%=================================================================================
%FORMATACAO ESTILO CAPITULOS
%=================================================================================

\geometry{textwidth=165.mm,left=30.0mm,right=20.0mm,top=25.0mm,bottom=20.0mm}
\linespread{1.5} % distância entre linhas
\parskip8pt      % da distância entre parágrafos
\titleformat{\chapter}[display]
{\normalsize\filcenter\sffamily}%\bfseries}
{\Large{\vskip -1.2in\sffamily\chaptertitlename} \thechapter\vskip -0.4in } {0pc} {\Large}

\normalsize
\titleformat{\section}%[display]
{\sffamily}{\thesection\hspace{0.08in}}{0pc}{\large}

\normalsize
\titleformat{\subsection}%[display]
{\sffamily}%{\bfseries}
{\thesubsection\hspace{0.08in}}{0pc}{\large}

%=================================================================================
% FORMATACAO CITAÇÃO MODELO ABNT2
%=================================================================================
% ---
% Configurações do pacote backref
% Usado sem a opção hyperpageref de backref
\renewcommand{\backrefpagesname}{Citado na(s) página(s):~}
% Texto padrão antes do número das páginas
\renewcommand{\backref}{}
% Define os textos da citação
\renewcommand*{\backrefalt}[4]{
\ifcase #1 %
Nenhuma citação no texto.%
\or
Citado na página #2.%
\else
Citado #1 vezes nas páginas #2.%
\fi}%
%=============================================================================================
% FORMATAÇÃO COR
%=============================================================================================
\hypersetup{
colorlinks,
linkcolor={blue!100!black},
citecolor={blue!100!black},
urlcolor={blue!100!black}
}

%================================================================================
\begin{document}
%============================================================================
\fontfamily{phv}\selectfont

%============================================================================
% inserir elementos pre-textuais
%============================================================================
%\pagenumbering{roman} % Numeração romana nas páginas iniciais
%\input{pre_textual/1_capa}

%\newpage
%==============================================================
% inserir sumário
%=================================================================
\renewcommand*\contentsname{Sumário}
\pdfbookmark[0]{\contentsname}{toc}
\tableofcontents
%=================================================================
% inserir lista de tabelas
%=================================================================
%\thispagestyle{empty}
%\listoftables 
%=================================================================
% inserir lista de figuras
%=================================================================
%\listoffigures


%=================================================================
% inserir resumos
%=================================================================



%============================================================================
% ELEMENTOS TEXTUAIS
%============================================================================

%======================================
% inserir formatação
%======================================
\thispagestyle{empty}
\pagenumbering{arabic}
\setcounter{page}{1}
%\input{textual/capitulo1}
	\chapter{Introdução}\label{capitulo1}

"No mundo hiperconectado em que vivemos, o WhatsApp (que vou chamar aqui de \textbf{WA}) se tornou uma extensão quase natural do nosso cotidiano. Seja para enviar mensagens rápidas, compartilhar memes, organizar encontros ou até mesmo discutir assuntos importantes, o aplicativo está sempre presente. Mas o que acontece quando decidimos nos desconectar dessa ferramenta tão onipresente? Movida pela curiosidade e por uma necessidade de 'desintoxicação digital', decidi embarcar em um experimento pessoal: ficar uma semana sem WA. Este ensaio relata os desafios, descobertas e reflexões dessa jornada, que, embora curta, revelou insights surpreendentes sobre minha relação com a tecnologia e as interações sociais."

Esse trecho acima foi escrito por uma inteligência artificial com um dos melhores custos-benefícios atualmente. Ouvi dizer que ela só consegue ser tão acessível porque foi treinada com base em modelos de IA mais robustos e caros. Já que estou aqui citando o amigo que me passou essa informação, vou tentar explicar, sem consultar uma IA, como funciona o treinamento de uma inteligência artificial.

Um neurônio artificial pode ser imaginado como uma unidade básica de processamento, composta por uma entrada e um peso. Imagine uma bola com um risco: a bola representa a entrada de dados, e o risco simboliza o peso atribuído a essa entrada. A informação recebida pela "bola" percorre um caminho determinado pelos pesos, que influenciam como a informação é processada. Um conjunto de neurônios interconectados forma uma rede neural, que é a base de uma inteligência artificial. Essas redes criam caminhos para a informação, ajustando os pesos ao longo do tempo para melhorar a precisão das decisões.

O treinamento de uma IA envolve a alimentação de grandes volumes de dados de entrada, que são usados para ajustar os pesos dos neurônios. Quanto mais variados e representativos forem os dados de treinamento, mais capaz a IA se torna de generalizar e tomar decisões precisas em diferentes cenários. Dizer que uma IA "cara" treinou uma IA "barata" significa que os dados ou modelos gerados pela primeira foram usados para treinar a segunda. Isso não é um paradoxo, mas sim uma prática comum na área de machine learning, conhecida como transfer learning (aprendizado por transferência), onde modelos pré-treinados são adaptados para novas tarefas.

Apesar de vivermos em uma era de complexidade tecnológica, quero criar uma brecha intencional para voltar minha mente a um estado mais pacífico, simples, silencioso, criativo e presente. Afinal, diante do complexo, é valioso recorrer ao simples.

Encontrei uma maneira de fazer isso: desinstalei o principal aplicativo de envio de mensagens atualmente e, de forma contrastante, começo a introdução deste relatório com um texto escrito por uma inteligência artificial.

Enfim, parece que todo movimento é em vão, mas espero colocar um pouco de ordem nas minhas ideias e criar um espaço propício para um encontro comigo mesma a partir dessa experiência.

O intuito, em um primeiro momento, é criar um relatório diário sobre a experiência de desconectar por uma semana. As regras são as seguintes:
\begin{itemize}
	\item Desinstalar o aplicativo WA do meu celular;
	\item Posso utilizar outros aplicativos para me comunicar, desde que não sejam da Meta | Social Metaverse Company. Não quero pesar o clima, mas o CEO dessa empresa fez uma saudação nazista publicamente;
	\item O item acima entra em vigor a partir do segundo dia de experiência, para casos de contato urgente;
	\item Reinstalar o WA apenas após sete dias da desinstalação.
\end{itemize}

Finalizado, este relatório se transformará em contos. A expectativa é escrever sete, mas vamos ver o que é possível de fato realizar. % Inclui o capítulo de introdução
	\chapter{Metodologia}\label{capitulo2}
Para garantir que o experimento fosse bem-sucedido, estabeleci algumas regras básicas:
\begin{enumerate}
	\item \textbf{Desinstalação do Aplicativo:} No primeiro dia, removi o WhatsApp do meu celular para evitar tentações.
	\item \textbf{Comunicação Alternativa:} Informei amigos e familiares sobre minha "ausência digital" e combinei outras formas de contato, como ligações ou e-mails, sms ou um outro aplicativo de mensagens.
	\item \textbf{Registro Diário:} Mantive um diário para anotar sentimentos, dificuldades e observações ao longo da semana.
\end{enumerate} % Inclui o capítulo de metodologia
	\chapter{Quarta-feira de cinzas}\label{capitulo3}

Surge uma agonia, uma vontade de não ser encontrada. Pensei a algum tempo de realizar um retiro de silêncio, mas refleti um pouco e percebi que um movimento desse tipo no âmbito digital poderia causar um bom impacto gastando bem menos, então resolvi seguir dessa forma

Expliquei a dois ou três amigos que o contato seria de outra forma nos próximos dias e em seguida desinstalei o aplicativo.

A ideia é lidar melhor com meus fantasmas, na verdade de ter mais tempo de lidar com eles. Tenho a impressão que na dinâmica moderna atual não sou bem eu que decido as coisas. Digo isso porque com a popularização dos algorítimos de inteligência artificial o marketing e propagandas são muito bem direcionados e não sei se minhas escolhas são frutos de uma necessidade interna ou de sugestões de propagandas de um perfil bem modelado pelos aplicativos do vale do silício.

Mas o fato é que avise que sairia do whatsApp e em seguida já exclui o aplicativo. Minha mãe que mora em uma cidade do interior onde a tecnologia assim como o tempo demoram a andar achou muito estranho quando eu, de supetão dei a notícia da ausência.

Ela tentou escrever tirando algumas dúvidas, mas eu já havia desinstalado o aplicativo e por conta disso não recebia mais as mensagens enviadas por lá.

Ela e minha irmã entenderam que eu poderia ter sido sequestrada e que aquela mensagem quem teria escrito era o próprio sequestrador, falando de telegram e palavras desconhecidas.

Depois de algum tempo mas não muito, quando voltei a ver o celular, já tinham umas 15 chamadas perdidas da minha irmã. Retornei e me dei conta do problema de comunicação que já tinha ocorrido no primeiro minuto de experiência. Liguei para minha em seguida para acalmá-la. 

Ela não acreditava que quem tava falando com ela era eu mesma e exigiu uma chamada de vídeo, funcionalidade disponibilizada pelo WhatsApp... Tal dinâmica de conversa deixou claro para mim que ela não entendia o que é um aplicativo e o que é um sistema operacional e muito menos tinha ideia da interindependência das coisas. Compartilhei um link de reunião pelo aplicativo de email e mandei para elas.

Me parece muito estranho ter que explicar que o normal é não usar um aplicativo ao invés de usá-lo mas ficou acordado que tenho que voltar a usar o aplicativo excluído depois de passar uma semana. 
	\chapter{Palestra Marcia Baja}\label{capitulo4}

Foi um dia estranho, parecia ter tempo demais que não sabia bem como administrar. Me propus a criar uma agenda que não criei. Fumei, continuo com a sensação de estar, de alguma maneira, atrasada. Por isso talvez esteja tão nervosa ultimamente qualquer coisa que saia do cronograma vai me atrasar mais e ai acabo ligando a cobra que ataca.
Acho que penso que tenho que dar conta de muita coisa e que essas coisas sequer existem. Por isso ficar sem o aplicativo dominante de troca de mensagens do Brasil hoje.
Quero conseguir voltar a contemplar. A mente tá sempre cheia de ideias, me lembro com saudades do silêncio.

Apesar das dificuldades eu consegui meditar durante uma palestra muito preciosa da Márcia Baja. Uma meditação sem  postura formal durante toda a prática e ainda com bastante movimentos e incômodos. 

A meditação acabou refletindo em uma yoga, fiz algumas saudações ao sol, alguns alongamentos, coloquei o pijama e fui dormir otimista. Com a sensação de que na sexta eu conseguiria em fim conquistar o mundo.

Sensações predominantes durante o dia: Desanimo, necessidade de organização p sair do marasmo, fuga para os ensinamentos como refugio.


\href{https://www.youtube.com/watch?v=WCxLdFbKu0w}{Palestra Marcia Baja no youtube}
	\chapter{Reviravoltas}\label{capitulo 5}

Acordei pronta para meu dia. Meditei 10 minutos. Tomei café, banho e comecei a trabalhar em um horário bem adequado.

Terminei uma sessão do curso de gerenciamento de memória, mais especificamente vi sobre lists e LinkedList \colorbox{yellow}{melhorar aq dando mais detalhes depois?}

perdi um pouco de tempo no youtube, fumei e foi ai que dia desandou. Agora cá estou eu as 2 da manhã escrevendo texto em uma sexta-feira a noite.

Acho que meu maior erro foi não ter meditado na hora do almoço e acabar fumando ai acabei indo correr na hora de almoçar. Tomei um banho gelado bem revigorante, só que na hora que eu fui comer acabei comendo demais e me bateu um sono daqueles incontroláveis logo após o almoço. Acabei cedendo. deitei um pouco ouvindo uma meditação, quando acordei ao invés de ir dar conta da vida peguei o celular entrei no instagram web, vi que um amigo tinha enviado umas fotos legais que precisavam de aplicativo para baixar ai instalei o aplicativo e fui absorvida pelo buraco negro da internet. Já vou desinstalar novamente.

 Amanhã vou acordar em um horário razoável e vou escalar

Sensações predominantes durante o dia: Otimismo, gratidão


	\input{textual/z-resultados} % Inclui o capítulo de resultados
	\input{textual/3-conclusao} % Inclui o capítulo de conclusão
	\input{textual/z-proximos_passos} % Inclui o capítulo de próximos passos

%\input{textual/capitulo5}
%\include{apendice}

%\bibliography{referencias}
%	\include{referencias}

	

	
\end{document}