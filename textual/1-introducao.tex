\chapter{Introdução}\label{capitulo1}

"No mundo hiperconectado em que vivemos, o WhatsApp (que vou chamar aqui de \textbf{WA}) se tornou uma extensão quase natural do nosso cotidiano. Seja para enviar mensagens rápidas, compartilhar memes, organizar encontros ou até mesmo discutir assuntos importantes, o aplicativo está sempre presente. Mas o que acontece quando decidimos nos desconectar dessa ferramenta tão onipresente? Movida pela curiosidade e por uma necessidade de 'desintoxicação digital', decidi embarcar em um experimento pessoal: ficar uma semana sem WA. Este ensaio relata os desafios, descobertas e reflexões dessa jornada, que, embora curta, revelou insights surpreendentes sobre minha relação com a tecnologia e as interações sociais."

Esse trecho acima foi escrito por uma inteligência artificial com um dos melhores custos-benefícios atualmente. Ouvi dizer que ela só consegue ser tão acessível porque foi treinada com base em modelos de IA mais robustos e caros. Já que estou aqui citando o amigo que me passou essa informação, vou tentar explicar, sem consultar uma IA, como funciona o treinamento de uma inteligência artificial.

Um neurônio artificial pode ser imaginado como uma unidade básica de processamento, composta por uma entrada e um peso. Imagine uma bola com um risco: a bola representa a entrada de dados, e o risco simboliza o peso atribuído a essa entrada. A informação recebida pela "bola" percorre um caminho determinado pelos pesos, que influenciam como a informação é processada. Um conjunto de neurônios interconectados forma uma rede neural, que é a base de uma inteligência artificial. Essas redes criam caminhos para a informação, ajustando os pesos ao longo do tempo para melhorar a precisão das decisões.

O treinamento de uma IA envolve a alimentação de grandes volumes de dados de entrada, que são usados para ajustar os pesos dos neurônios. Quanto mais variados e representativos forem os dados de treinamento, mais capaz a IA se torna de generalizar e tomar decisões precisas em diferentes cenários. Dizer que uma IA "cara" treinou uma IA "barata" significa que os dados ou modelos gerados pela primeira foram usados para treinar a segunda. Isso não é um paradoxo, mas sim uma prática comum na área de machine learning, conhecida como transfer learning (aprendizado por transferência), onde modelos pré-treinados são adaptados para novas tarefas.

Apesar de vivermos em uma era de complexidade tecnológica, quero criar uma brecha intencional para voltar minha mente a um estado mais pacífico, simples, silencioso, criativo e presente. Afinal, diante do complexo, é valioso recorrer ao simples.

Encontrei uma maneira de fazer isso: desinstalei o principal aplicativo de envio de mensagens atualmente e, de forma contrastante, começo a introdução deste relatório com um texto escrito por uma inteligência artificial.

Enfim, parece que todo movimento é em vão, mas espero colocar um pouco de ordem nas minhas ideias e criar um espaço propício para um encontro comigo mesma a partir dessa experiência.

O intuito, em um primeiro momento, é criar um relatório diário sobre a experiência de desconectar por uma semana. As regras são as seguintes:
\begin{itemize}
	\item Desinstalar o aplicativo WA do meu celular;
	\item Posso utilizar outros aplicativos para me comunicar, desde que não sejam da Meta | Social Metaverse Company. Não quero pesar o clima, mas o CEO dessa empresa fez uma saudação nazista publicamente;
	\item O item acima entra em vigor a partir do segundo dia de experiência, para casos de contato urgente;
	\item Reinstalar o WA apenas após sete dias da desinstalação.
\end{itemize}

Finalizado, este relatório se transformará em contos. A expectativa é escrever sete, mas vamos ver o que é possível de fato realizar.