\chapter{O Peso do Tempo}\label{capitulo3}

Depois do susto inicial, consegui fazer um acordo com minha família sobre esse projeto. Quando se está longe de casa, não é possível se isolar por muito tempo — pessoas queridas ficam preocupadas. Por isso, o limite de uma semana para a execução desse experimento. Acordo feito, limite definido. Minha mãe e irmã foram embora, retomando suas rotinas, e eu fiquei novamente sozinha comigo mesma.

O dia parecia estranho. A sensação era de que o tempo havia se esticado, como se os minutos estivessem ganhado uma elasticidade desconhecida. Sem o WhatsApp, aquele fluxo constante de mensagens e notificações que costumava preencher cada intervalo do meu dia simplesmente desapareceu. De repente, me vi diante de um vazio que não sabia bem como administrar.

Me propus a criar uma agenda, mas ela só existiu na minha cabeça. Fiquei paralisada, olhando para folhas em branco, enquanto a ansiedade sussurrava no meu ouvido:
- Você está atrasada.

Atrasada para quê? Nem eu sabia direito. Talvez para a vida, para as expectativas que eu mesma criei, para aquela lista invisível de tarefas que nunca acaba. Qualquer coisa que saísse do imaginado parecia me empurrar ainda mais para trás, e eu me via encurralada pela cobrança interna. A cada intervalo de tempo, eu descia mais fundo em uma realidade que despertava em mim sensações de medo e fuga.

Fumei um cigarro atrás do outro, tentando acalmar os nervos. A fumaça subia em espirais, levando com ela um pouco da minha inquietação, mas não o suficiente. A sensação de estar sempre correndo contra o tempo continuava — e piorava. Antes, eu sabia, ou tinha a impressão de saber, o que as outras pessoas estavam fazendo e conseguia avaliar o tamanho da minha desordem pessoal. Agora, sem essa referência, minha ansiedade era totalmente abstrata, como se eu estivesse em uma maratona sem linha de chegada. 

Minhas obsessões afloravam pois não havia um objetivo claro ao qual elas pudessem se apegar ou algo que pudesse me distrair ao ponto de não enxergá-las. No meio disso tudo, uma pergunta ecoava na minha cabeça: por que sinto que tenho que dar conta de tanta coisa?

Ficar sem esse aplicativo foi como tirar um peso das costas, mas também como perder uma muleta. Sem ele, percebi o quanto dependia dessa ferramenta não só para me comunicar, mas também para me distrair, para preencher os vazios que eu não queria encarar. E agora esses vazios estavam ali, expostos, pedindo atenção, gritando, me perturbando, como a companhia forçada de alguém com quem não gostaria de estar. Era eu encarando o abismo.

Quero conseguir voltar a contemplar. Já tive momentos em que conseguia parar e simplesmente existir, sem pressa ou cobranças. Sentir meu chakra cardíaco pulsar, a energia do amor passando por mim como se eu fosse um veículo sorridente. Agora, eu mesma lendo isso acabo rindo, porque parece mentira. De fato, acredito que a pandemia, o capitalismo e Brasília tenham me endurecido.

Meu refúgio do medo, desde pequena, sempre foram os estudos. Com um caderno, livro ou computador, eu criava minha bolha particular, que sempre me protegeu. Dentro dela se manifesta a paz necessária para equilibrar minha energia. Resolvi dessa vez estudar filosofia para criar esse espaço seguro, mais especificamente olhei para os ensinamentos budistas.

Consegui meditar um pouco, mas foi uma meditação diferente, sem postura formal, sem a rigidez de um zafu ou de um altar. Eu estava deitada nos colchões que uso para escalar, me mexendo, ajustando a posição, sentindo incômodos aqui e ali. Mas, mesmo assim, consegui me conectar com aquele momento.

A meditação acabou refletindo em uma prática de yoga. Fiz algumas saudações ao sol, alonguei o corpo, senti os músculos despertarem e a mente se acalmar. Coloquei o pijama e fui dormir com um sentimento raro ultimamente: otimismo. A sensação de que, na sexta-feira, eu finalmente conseguiria conquistar o mundo. Ou, pelo menos, o meu mundo.

As sensações predominantes durante o dia foram de desânimo, a necessidade de organização para sair do marasmo e a fuga para os ensinamentos como refúgio. Foi um dia de altos e baixos, de confrontos internos e pequenas conquistas. No fim, tive a sensação de que essa proposta de desconectar acabaria me levar a lugares desafiadores de uma forma que nem imaginava.

\section{Meditação em Ação}

Resolvi, portanto, fazer amizade com minha inquietação. Já que ela está comigo todos os dias é melhor que comecemos a nos dar bem. Mas afinal, como posso transformar minha vida, que possui essa aparência tão simples e corriqueira em um caminho de realização pessoal?

Às vezes, penso que preciso me mudar e ir para um lugar afastado. É muito bom estar na natureza; minha saúde, mente e coração melhoram. Acontece que esse movimento não é algo simples de ser feito. Muito por conta de compromissos já estabelecidos, como trabalho, casa e amigos que residem aqui. Então, será que é possível, exatamente no lugar onde estou, fazendo as coisas que devem ser feitas, com as pessoas que me cercam, com as comidas que como, sem fazer nada radical, ressignificar meu espaço?

Lugares e pessoas especiais podem muito bem ser uma ilusão enquanto não entender alguns aspectos sutis internos que me conecta às experiências externas. Essas sutilezas internas de emoções e sensações me acompanham de forma incessante não é possível separar. Se eu for morar em algum pico de escalada, meu mundo interno vai junto. Se for para fora do Brasil, também. Se ficar aqui, a mesma coisa. Então, não existe um lugar perfeito onde, enfim, eu vou ser feliz e tudo vai dar certo, logo não preciso ficar aflita para chegar em lugar nenhum.

Falar de mundo interno e externo me faz imaginar que existe uma coisinha dentro de mim e que vou derramando esses significados devagarzinho para o lado de fora, construindo assim o mundo externo. Mas a realidade é que esses dois lugares existem simultaneamente. Só existe um mundo, que é um lugar sutil onde habito o tempo todo, onde mora minha consciência e é onde tudo acontece. Logo, eu sou responsável pelo que sinto, pelo que estou nutrindo e pelo que vejo. 

Ao começar a trabalhar o olhar que meus olhos lançam para o mundo todas as pessoas passam a ser importantes. É a interação com o outro que sinaliza algo dentro de mim. Se eu sentir raiva, não é culpa de uma pessoa irritante, mas sim porque tenho uma região delicada dentro de mim que ao entrar em contato com determinados comportamentos e movimentos dessa pessoa ficam exposto e acessível me deixando com raiva. Ao organizar a cozinha ou sala, na verdade, estou organizando meu próprio mundo interno. 

A prática da meditação entra como ferramenta para familiarização com uma base de silêncio e apoio que viabiliza uma movimentação pacífica no mundo mais. 



Dessa forma não faz sentido discutir quando alguém vem dizer que o mundo é assim ou assado, porque, para aquela pessoa, o mundo é aquilo mesmo, e para mim é diferente. Pois cada ser possui uma base interna gerada a partir de suas experiências e forma de vida.

Por exemplo, ninguém te obrigou a ler isso. Não existem amarras na cadeira, presilhas nos olhos e alguém te forçando dizendo: "Leia isso! Chegou sua hora, por favor." Ninguém fez isso. Você está nessa leitura por livre e espontânea vontade. Então, o caminho só funciona se a gente quiser caminhar.

Quanto logo entendermos esse mundo interno, é possível se libertar esses nós vindo de padrões e comportamentos . Isso está ao nosso alcance. Não é necessário remédios, cirurgias ou trocar de cérebro. Nada. Apenas sentar em silêncio um pouco e olhar para dentro.

Todos nós estamos vivendo dentro da mente, e isso nem quer dizer que é da nossa mente — é dentro da mente sutil do mundo. Quando alimento padrões mentais negativos, passo a comunicar isso para as pessoas, às vezes até sem nenhuma palavra. Se alguém aparece todo crispado tipo um gato por dentro, magicamente quem estiver ao redor também começa a bater cadeira, prato, pé, passam mal, ficam com dor de cabeça. Às vezes, a pessoa nem entende o que está acontecendo. Ela pensa: "Nossa, eu estava tão bem, agora estou começando a ficar com raiva das coisas." Isso por conta dos sinais sutis e grosseiros que captamos e emitimos. Pode ser que a gente chegue irritado em um lugar e não fale nada, mas aquele nosso bico comunica.

Ao colocar energia em realizar um trabalho interno de liberar nossas emoções perturbadas e esses mundos mentais negativos, começamos a nos transformar em espelho de qualidades para os outras. Essa é a melhor maneira de mudar as pessoas: nos tornarmos espelhos dessas qualidades que queremos ver no outro.

Nossa mente possui um estado padrão conhecido como base primordial a partir da qual vários outros tipos de mentes podem surgir. Por exemplo: Quando nos familiarizamos com formas de sentir, pensar, ser e nos comportar, aquilo vai ficando marcado e nos endurecendo em um determinado lugar e forma de pensar criando uma posição de mente ao qual estamos habituados que é limitada, condicionada, conectada a experiências negativas que nós perturbam e nos faz acreditar que essa condição é tudo que existe. Pela meditação, nossa mente pode descansar e formatar o estado das coisas. 

Então esse é o trabalho: Descansar de nós mesmos.

\href{https://www.youtube.com/watch?v=WCxLdFbKu0w}{Palestra Marcia Baja no YouTube}

