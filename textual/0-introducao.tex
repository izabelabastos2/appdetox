\chapter{Introdução}\label{capitulo1}
"No mundo hiperconectado em que vivemos, o WhatsApp se tornou uma extensão quase natural do nosso cotidiano. Seja para enviar mensagens rápidas, compartilhar memes, organizar encontros ou até mesmo discutir assuntos importantes, o aplicativo está sempre presente. Mas o que acontece quando decidimos nos desconectar dessa ferramenta tão onipresente? Movida pela curiosidade e por uma necessidade de "desintoxicação digital", decidi embarcar em um experimento pessoal: ficar uma semana sem WhatsApp (que vou chamar aqui de \boldmath{WA}). Este ensaio relata os desafios, descobertas e reflexões dessa jornada, que, embora curta, revelou insights surpreendentes sobre minha relação com a tecnologia e as interações sociais."

Esse trecho acima foi escrito pela inteligência artificial com melhor desempenho em relação ao custo atualmente. Eu ouvi dizer que ela só consegue ser barata assim porque ela usou a inteligência artificial cara para treinar. E já que estou aqui plagiando o amigo que me passou essa informação , que não é plagio mais porque acabo de passar a referência, vou explicar como funciona a inteligência de uma AI sem pesquisar isso em uma AI.

Um neurônio é definido com uma entrada e um peso, da para imaginar uma bola com um risco onde a bola é a entrada e o risco é o peso. A bola sempre recebe uma informação que posteriormente irá percorrer um caminho de menor peso. Um conjunto de neurônios é uma inteligência artificial que cria caminhos que custam um peso por onde a informação passa até chegar em um objetivo.

O conjunto de informações de entrada fornecida aos nerônios inicialmente configura um cenário de treinamento. A partir da variação nos dados de entrada a AI fica mais esperta porque ela consegue considerar mais cenários antes de tomar uma decisão.

Então dizer que a AI cara treinou a AI barata significa que os dados de entrada que alimentaram o treinamento da AI barata foram geradas por uma AI cara então a AI barata só existiria porque a cara existe. Temos ai um paradoxo moderno.


Apesar da estamos vivos nesse tempo quero tentar criar uma brecha intencional para voltar minha mente para um estado pacífico, simples, silencioso, criativo, presente, semeável. Afinal, diante do complexo é valioso usar o simples.

Encontrei uma maneira de fazer isso, desinstalei o principal aplicativo de envio de mensagens atualmente e de forma contrastante começo a introdução do meu relatório dessa experiência com um texto escrito por uma inteligencia artificial.

Enfim

Parece mesmo que todo movimento é em vão, mas eu espero colocar um pouco de ordem nas minhas ideias e criar um espaço propício para um encontro comigo mesma a partir dessa experiência.


O intuito em primeiro momento é criar um relatório diário sobre a experiência de desconectar por uma semana. As regras são as seguintes:

\item {desinstalar o aplicativo WA do meu celular}
\item {posso utilizar outros aplicativos para me comunicar desde que não sejam da Meta | Social Metaverse Company. Eu não gostaria de pesar o clima mas o CEO dessa iniciativa fez saudação nazista publicamente. }
\item {o item acima entra em vigor a partir do dia dois de experiência para caso de contato urgente}
\item {reinstalar o WA apenas depois de sete dias a partir da desinstalação}

Finalizado, esse relatório irá virar contos. A expectativa é escrever sete, mas vamos ver o que é possível de fato de ser feito.