\chapter{Quarta-feira de cinzas}\label{capitulo3}

Surge uma agonia, uma vontade de não ser encontrada. Pensei a algum tempo de realizar um retiro de silêncio, mas refleti um pouco e percebi que um movimento desse tipo no âmbito digital poderia causar um bom impacto gastando bem menos, então resolvi seguir dessa forma

Expliquei a dois ou três amigos que o contato seria de outra forma nos próximos dias e em seguida desinstalei o aplicativo.

A ideia é lidar melhor com meus fantasmas, na verdade de ter mais tempo de lidar com eles. Tenho a impressão que na dinâmica moderna atual não sou bem eu que decido as coisas. Digo isso porque com a popularização dos algorítimos de inteligência artificial o marketing e propagandas são muito bem direcionados e não sei se minhas escolhas são frutos de uma necessidade interna ou de sugestões de propagandas de um perfil bem modelado pelos aplicativos do vale do silício.

Mas o fato é que avise que sairia do whatsApp e em seguida já exclui o aplicativo. Minha mãe que mora em uma cidade do interior onde a tecnologia assim como o tempo demoram a andar achou muito estranho quando eu, de supetão dei a notícia da ausência.

Ela tentou escrever tirando algumas dúvidas, mas eu já havia desinstalado o aplicativo e por conta disso não recebia mais as mensagens enviadas por lá.

Ela e minha irmã entenderam que eu poderia ter sido sequestrada e que aquela mensagem quem teria escrito era o próprio sequestrador, falando de telegram e palavras desconhecidas.

Depois de algum tempo mas não muito, quando voltei a ver o celular, já tinham umas 15 chamadas perdidas da minha irmã. Retornei e me dei conta do problema de comunicação que já tinha ocorrido no primeiro minuto de experiência. Liguei para minha em seguida para acalmá-la. 

Ela não acreditava que quem tava falando com ela era eu mesma e exigiu uma chamada de vídeo, funcionalidade disponibilizada pelo WhatsApp... Tal dinâmica de conversa deixou claro para mim que ela não entendia o que é um aplicativo e o que é um sistema operacional e muito menos tinha ideia da interindependência das coisas. Compartilhei um link de reunião pelo aplicativo de email e mandei para elas.

Me parece muito estranho ter que explicar que o normal é não usar um aplicativo ao invés de usá-lo mas ficou acordado que tenho que voltar a usar o aplicativo excluído depois de passar uma semana. 