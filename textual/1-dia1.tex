\chapter{Quarta-feira de cinzas}\label{capitulo3}

Existe o completo vazio no espaço. Condições ideias de temperatura e pressão e no meio do silêncio surge uma colisão que gerou vida. No meu caso é diferente, minha cabeça nunca está vazia, pelo contrário, é uma grande coleção de ideias, modelos, projeções. No meio do caos que rege minha vida da mesma forma que orquestra todo universo acontece uma colisão. Mais especificamente: Uma colisão de ideias.

Todo esse movimento interno refletiu em uma sequência de movimentos externos: Avisei alguns amigos e família sobre a exclusão do aplicativo de mensagens e exclui.

Inicialmente não me parecia grande coisa, fui tomar um banho na esperança de que a água me ajudaria a processar os acontecimentos dos últimos dias. Afinal foi carnaval no Brasil, as coisas andam meio bagunçadas no meu emocional e ando na busca de entender melhor a realidade para tomar minhas decisões de forma assertiva.

Ainda não entendo porque me cobro dessa maneira, toda vez que consigo criar um modelo ideal da minha vida eu acabo por me repousar em uma sensação de controle tão prazerosa quanto curta. A vida sempre me surpreende e carregar tudo para outras direções. Em algumas coisas eu acerto, mas só naquelas que envolvem muita burocracia, mesmo assim as vezes eu erro.

Com isso em mente eu procuro manter um estado aberto para as surpresas que vão aparecer por conta de um movimento inusitado, afinal quem fica sem aplicativo hoje em dia além de mim e da Palinda, uma amiga queridíssima e escassa hoje em dia, daquelas lindas e sem WA. O "Pa" é porque o nome real dela é um palíndromo. 

Palíndromo é um problema clássico de programação, trata-se de criar um esquema lógico em que dado uma sequencia de caracteres fornecida, verifica se a sequência inicial é igual a sequência invertida. Como o que acontece em ovo, osso e com os números 121, 4554, 378873. A parte do linda é auto explicativa. Ela é uma amiga da escalada que também é amiga de estudos. Eu olho para as pessoas que estão perto de mim hoje e sinto tanta gratidão que alguma coisa eu devo estar fazendo certo ao fim das contas.

Sai do banho sem saber direito o que fazer já que oficialmente eu teria que realocar todo o tempo que estava acostumada a empregar em existir na rede elétrica, mesmo que isso implicasse em existir na rede elétrica de uma outra forma a partir de outros aplicativos. Me recusei, até vi meu celular no canto da sala e me desviei, peguei tintas e pincéis, papeis e empreguei minha energia em ser criativa de uma forma aberta. sem limite de tempo me segurando, sem mensagens para responder. Apenas eu comigo mesma em um estúdio em Brasília.






O que me levou até meu celular e me dei conta que minha irmã tinha me ligado repetidamente nos últimos 30 minutos, retornei.






*******************************************************************************************************************************************



 a algum tempo de realizar um retiro de silêncio, mas refleti um pouco e percebi que um movimento desse tipo no âmbito digital poderia causar um bom impacto gastando bem menos, então resolvi seguir dessa forma

Expliquei a dois ou três amigos que o contato seria de outra forma nos próximos dias e em seguida desinstalei o aplicativo.

A ideia é lidar melhor com meus fantasmas, na verdade de ter mais tempo de lidar com eles. Tenho a impressão que na dinâmica moderna atual não sou bem eu que decido as coisas. Digo isso porque com a popularização dos algorítimos de inteligência artificial o marketing e propagandas são muito bem direcionados e não sei se minhas escolhas são frutos de uma necessidade interna ou de sugestões de propagandas de um perfil bem modelado pelos aplicativos do vale do silício.

Mas o fato é que avise que sairia do whatsApp e em seguida já exclui o aplicativo. Minha mãe que mora em uma cidade do interior onde a tecnologia assim como o tempo demoram a andar achou muito estranho quando eu, de supetão dei a notícia da ausência.

Ela tentou escrever tirando algumas dúvidas, mas eu já havia desinstalado o aplicativo e por conta disso não recebia mais as mensagens enviadas por lá.

Ela e minha irmã entenderam que eu poderia ter sido sequestrada e que aquela mensagem quem teria escrito era o próprio sequestrador, falando de telegram e palavras desconhecidas.

Depois de algum tempo mas não muito, quando voltei a ver o celular, já tinham umas 15 chamadas perdidas da minha irmã. Retornei e me dei conta do problema de comunicação que já tinha ocorrido no primeiro minuto de experiência. Liguei para minha em seguida para acalmá-la. 

Ela não acreditava que quem tava falando com ela era eu mesma e exigiu uma chamada de vídeo, funcionalidade disponibilizada pelo WhatsApp... Tal dinâmica de conversa deixou claro para mim que ela não entendia o que é um aplicativo e o que é um sistema operacional e muito menos tinha ideia da interindependência das coisas. Compartilhei um link de reunião pelo aplicativo de email e mandei para elas.

Me parece muito estranho ter que explicar que o normal é não usar um aplicativo ao invés de usá-lo mas ficou acordado que tenho que voltar a usar o aplicativo excluído depois de passar uma semana. 