\chapter{Quarta-feira de cinzas}\label{capitulo3}

Existe um vazio quase completo no espaço, o vácuo. Isso significa que a pressão é extremamente baixa, próxima de zero, e tal efeito resulta na ausência de atmosfera. Já a temperatura varia drasticamente dependendo da proximidade de fontes de calor, como estrelas, por exemplo. Sempre achei muito curioso que a temperatura se propague no espaço mesmo com a ausência de atmosfera. Nesse caso, a propagação ocorre por meio de radiação, que é a transferência de calor através de ondas eletromagnéticas. Imagine só o silêncio completo... É assim no espaço, pois o som é uma onda mecânica que necessita de um meio material, como a atmosfera, para se propagar. No meio desse silêncio utópico, surge uma colisão completamente silenciosa que gerou as condições necessárias para a manifestação da vida. Acredita-se que a Lua tenha se formado a partir de uma colisão entre a Terra e um corpo do tamanho de Marte há bilhões de anos.

No meu caso particular, é diferente. Minha cabeça nunca está vazia. Pelo contrário, é uma grande coleção de ideias, modelos, projeções, caminhos, emoções e etc. Conforme foco a atenção em alguma emoção, a temperatura e a pressão do meu corpo mudam, e meu comportamento também, contribuindo ou prejudicando as experiências que vivencio. Em meio ao caos que rege minha vida, da mesma forma que orquestra todo o universo, acontece uma colisão muito longe de ser silenciosa. Há música das mais diversas, cheiros, sons, cores, formando uma coleção louca de significados e gerando, assim, a manifestação da minha personalidade. Todo esse movimento interno refletiu em uma sequência de movimentos externos: será que consigo influenciar mais ativamente na dinâmica desse caos que me permeia?

O primeiro passo para a resolução de um problema, na minha filosofia, é descobrir esse problema. Então, aplicando essa metodologia, comecei a procurar a causa principal que originou a perturbação ou interferência no meu tráfego de ideias. Refleti um pouco e optei pela interferência (impacto direto de agentes externos), assim fica mais fácil, porque tiro a culpa de mim e coloco em algo ou alguém. Agora, qual agente externo seria o responsável por todas essas ideias desgovernadas que venho tentando lidar? Resolvi culpar uma sociedade exacerbadamente informada que, ao meu ver, está extremamente dependente de aplicativos como o WhatsApp. Pronto, defini o meu vilão do momento, mas, como sou uma mulher da ciência, preciso provar minha teoria e optei por fazer isso em um modelo experimental.

Avisei alguns amigos e familiares sobre a exclusão do aplicativo de mensagens. Não prolonguei a conversa após o anúncio; simplesmente excluí o WhatsApp. Inicialmente, não me parecia grande coisa. Fui tomar um banho na esperança de que a água me ajudaria a processar os acontecimentos dos últimos dias. Afinal, foi Carnaval no Brasil, as coisas andam meio bagunçadas no meu emocional, e estou na busca de entender melhor a realidade para tomar melhores decisões de forma assertiva. Às vezes, penso que me cobro demais. Toda vez que consigo criar um modelo ideal da minha vida, acabo me repousando em uma sensação de controle tão prazerosa quanto curta, pois a vida conduz o fluxo de acontecimentos em direções completamente inesperadas. Em algumas coisas eu acerto, mas só naquelas que envolvem muita burocracia; mesmo assim, às vezes erro.

Com isso em mente, procuro sempre equilibrar dentro de mim um estado aberto para as surpresas que vão aparecer e é esse estado que quero nutrir durante essa experiência. Afinal, não conheço quem não utilize esse aplicativo além de mim e da Palin. O "Pa" é porque o nome real dela é um palíndromo. Palíndromo é um problema clássico de programação; trata-se de criar um esquema lógico em que, dada uma sequência de caracteres fornecida, verifica-se se a sequência inicial é igual à invertida. Como o que acontece em "ovo", "osso" e com os números 121, 4554, 378873. A parte do "lin" é autoexplicativa: ela é linda por dentro e por fora. Ela é uma amiga da escalada que também é amiga de estudos. Olho para as pessoas que estão perto de mim hoje e sinto tanta gratidão e orgulho que, ao fim das contas, alguma coisa devo estar fazendo certo.

Saí do banho sem saber direito o que fazer, já que, oficialmente, eu teria que realocar todo o tempo que estava acostumada a empregar em existir na rede elétrica, mesmo que isso implicasse em existir na rede elétrica de uma outra forma, a partir de outros aplicativos. Me recusei. Até vi meu celular no canto da sala, mas desviei a atenção. Finalmente, estava poderosa, com o controle em minhas mãos (imagine risadas maléficas, só que do bem). Peguei tintas, pincéis, papéis e empreguei minha energia em ser criativa de uma forma aberta. Sem me preocupar com o limite de tempo, sem mensagens para responder, sem satisfação a dar. Apenas eu comigo mesma em um estúdio em Brasília

Passei praticamente o dia todo no meu mundo particular quando escutei a campainha. Fui ver quem era à porta e dei de cara com minha mãe e irmã. Para entender melhor o peso dessas presenças aqui em casa, é necessário fornecer algumas informações. Minha mãe tem medo de voar, por isso geralmente sou eu que vou até ela. Já fazem dois anos que não a vejo por conta de alguns imprevistos da vida adulta e da logística. Minha mãe mora em uma cidade no interior do Espírito Santo onde para chegar é necessário pegar um voo de aproximadamente duas horas e depois um ônibus com uma viagem de aproximadamente quatro horas. Minha irmã mora em São Paulo e é uma pessoa super ocupada. Apesar de todos esses pontos e além do preço das passagens aéreas serem exorbitantes de última hora, pensei estar alucinando, sob efeito de abstinência das redes sociais, vendo aquela miragem da minha família comigo em Brasília.

Fiquei em choque por um momento, avaliando a situação, quando minha mãe começou a gritar, trazendo à tona lembranças da nossa dinâmica juntas: "- Cacaô, pelo amor da Deusa, minha filha, você quer matar sua mãe e sua irmã do coração!? Onde você estava com a cabeça de não responder a gente?"

Enquanto acalmava as minhas feras preferidas, ela me explicou seu raciocínio. Ao comunicar a exclusão do aplicativo e sair em seguida, ela achou que eu tinha sido raptada e que quem estava realmente escrevendo a mensagem era o sequestrador, que havia feito isso para ter tempo de fugir para longe comigo. Tenho alguns tios na Polícia Federal, pessoas que ela já havia acionado para empenhar esforços em me encontrar e que confirmaram meu endereço para elas.  É muito curioso o fato de que bastou excluir um único aplicativo e de algumas horas para algumas pessoas da Polícia Federal começarem a me procurar. Meu tio Orlando é delegado e achou razoável a teoria da minha mãe. Foi com a ajuda dele que ela conseguiu pegar um avião para Brasília. Minha irmã saiu de São Paulo sem precisar organizar uma logística complexa, pois essa mulher é ligeira e especialista em resolver B.O., já que é advogada. Vi meu celular e havia uma quantidade exorbitante de ligações das duas.


Tive que prestar contas à polícia, à minha família, mas me senti segura, amada e confusa. Achei legal essa movimentação tão rápida, mas também me pareceu muito estranho ter que explicar que o normal é não usar um aplicativo, em vez de usá-lo, afinal eu nasci antes do advento desse tipo de tecnologia. Ao menos, esse caos que surgiu do fato deu tentar controlar meu caos me proporcionou a satisfação de dormir de conchinha com as mulheres mais importantes do meu universo.
